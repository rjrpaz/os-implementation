\documentclass{article}

%\setlength{\oddsidemargin}{0.0 in}
%\setlength{\evensidemargin}{0.0 in}
%\setlength{\textwidth}{7.0 in}

\oddsidemargin 0cm
\topmargin -2.5cm     
\textwidth 16.5cm   
\textheight 24.0cm 

\def\titledquestion#1#2{\section{ {#1} ({#2} pts.)}}

\begin{document}
\begin{center}{\large 15-410, Spring 2011, Homework Assignment 1.} \\
%\Large{Due Wednesday, February 24, 20:59:59 p.m.}
%\Large{Due Monday, October 6, 20:59:59 p.m.}
%\Large{Due Tuesday, October 5, 20:59:59 p.m.}
\Large{Due Tuesday, February 22, 20:59:59}
\end{center}

% \begin{center}
% \fbox{{\Large You need} {\large complete only} Question~3 and one other question.}
% \end{center}

{\small Please} observe the {\large non-standard} {\Large submission} {\LARGE time}...
%{\Large Please} {\large observe the} non-standard {\small submission time}...
As we intend to make solutions available on the web site immediately
thereafter for exam-study purposes, please turn your solutions
in on time.

Homework must be submitted in either PostScript or PDF format (not:
Microsoft Word, Word Perfect, Apple Works, LaTeX, XyWrite, WordStar,
etc.).  Submit your answers by placing them in the appropriate
hand-in directory, e.g.,
{\tt /afs/cs.cmu.edu/academic/class/15410-s11-users/\$USER/hw1/\$USER.ps} or
{\tt /afs/cs.cmu.edu/academic/class/15410-s11-users/\$USER/hw1/\$USER.pdf}.
A plain text file (.text or .txt) is also acceptable, though it must conform to Unix
expectations,
meaning lines of no more than 120 characters separated by newline characters
(note that this is {\em not} the Windows convention or the MacOS convention).
Please avoid creative filenames such as {\tt hw1/my\_15-410\_homework.PdF}.

%%%%%%%%%%%%%%%%%%%%%%%%%%%%%%%%%%%%%%%%%%%%%%%%%%%%%%%%%%%%%%%%%%%%%%
\titledquestion{Tape drives}{4}

Consider a system with
  three processes and
  five tape drives.
The maximal needs of each process are declared below:

\begin{tabular}[b]{|l|c|l|c|l|}
\multicolumn{5}{c}{Resource Declarations} \\
\cline{1-1} \cline{3-3} \cline{5-5}
Process A && Process B && Process C \\
\cline{1-1} \cline{3-3} \cline{5-5}
3 tape drives && 2 tape drives && 4 tape drives \\
\cline{1-1} \cline{3-3} \cline{5-5}
\end{tabular}

Imagine the system is in the state depicted
below.  List one request which the system should grant right
away, and one request which the system should react to by
blocking the process making the request.  Briefly justify
each of your answers.

\begin{tabular}[b]{|l|r|r|r|}
\multicolumn{1}{c}{Who} &
 \multicolumn{1}{c}{Max} &
 \multicolumn{1}{c}{Has} &
 \multicolumn{1}{c}{Room} \\
\hline
A      & 3 & 1 & 2 \\ \hline
B      & 2 & 1 & 1 \\ \hline
C      & 4 & 2 & 2 \\ \hline
System & 5 & 1 & - \\ \hline
\end{tabular}

\vspace{2in}
\begin{center}
{\textbf{(Continued on next page)}}
\end{center}

\clearpage

%%%%%%%%%%%%%%%%%%%%%%%%%%%%%%%%%%%%%%%%%%%%%%%%%%%%%%%%%%%%%%%%%%%%%%
\titledquestion{Race conditions}{6}
%%%%%%%%%%%%%%%%%%%%%%%%%%%%%%%%%%%%%%%%%%%%%%%%%%%%%%%%%%%%%%%%%%%%%%

Consider the two threads whose body functions are shown below (imagine
each is invoked via \texttt{thr\_create()}):

\begin{verbatim}
int n = 1;

void *thread0(void *ignored) {
  while (n >= 0)
    --n;
  return (0);
}

void *thread1(void *ignored) {
  while (n < 1)
    ++n;
  return (0);
}

\end{verbatim}

\subsection{3 pts}
Show an execution trace in which thread 1's loop \textit{body}
executes exactly three times.

\subsection{3 pts}
Show an execution trace in which both thread 0 and thread 1
run forever.

When showing an execution trace, use the tabular trace format found in
the lecture slides.
Please be sure that your trace is clear and conclusive
(otherwise your answer will not receive full credit).
You may use more or fewer lines or columns than are provided
in this sample table.
You may change the column headings if you wish.

{\large
\begin{tabular}[b]{|r|c|c|}
\multicolumn{3}{c}{Execution Trace} \\
\hline
time & Thread~0 & Thread~1 \\
\hline
 0 & \hspace{1in} & \hspace{1in} \\
\hline
 1 & & \\
\hline
 2 & & \\
\hline
 3 & & \\
\hline
 4 & & \\
\hline
 5 & & \\
\hline
\end{tabular}
}

\end{document}
